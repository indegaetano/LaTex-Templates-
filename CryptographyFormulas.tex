\documentclass[11pt,addpoints]{exam}
\usepackage{amsfonts,amssymb,amsmath, amsthm}
\usepackage{graphicx}
\usepackage{systeme}
\usepackage{pgf,tikz,pgfplots}
\pgfplotsset{compat=1.15}
\usepgfplotslibrary{fillbetween}
\usepackage{mathrsfs}
\usetikzlibrary{arrows}
\usetikzlibrary{calc}
\usepackage{tikz}
\usepackage{multicol}
\usepackage{geometry}
\usepackage[left=1cm, right=1cm, top=1.5cm, bottom=1cm]{geometry}

\usepackage{array}
\usepackage{booktabs}
\newtheorem{definition}{Definition}


\pagestyle{headandfoot}

\firstpageheader{\textbf{Cryptography Formula Sheet}}{}{}
%\firstpageheadrule



\firstpagefooter{}{}{}
\runningfooter{}{}{}


\begin{document}

\begin{multicols}{2} % Creates a two-column layout
\subsection*{Plain Text Letters: }
\begin{center}
\begin{tabular}{||c c c c c c c c||} 
 \hline
 A & B & C & D & E & F & G & H  \\ [2ex] 
 \hline
 00 & 01 & 02 & 03 & 04 & 05 & 06 & 07  \\ [2ex] 
 \hline \hline
 I & J & K & L & M & N & O & P  \\ [2ex] 
 \hline
 08 & 09 & 10 & 11 & 12 & 13 & 14 & 15  \\ [2ex] 
 \hline \hline
 Q & R & S & T & U & V & W & X  \\ [2ex] 
 \hline
 16 & 17 & 18 & 19 & 20 & 21 & 22 & 23  \\ [2ex] 
 
 \hline \hline 
  Y & Z & \quad & \quad   &\quad  & \quad & \quad  &  \quad  \\ [2ex] 
 \hline
 24 & 25 & \quad  & \quad  & \quad  &  &  &   \\ [2ex] 
 \hline 

\end{tabular}
\end{center}

\subsection*{Binary Conversion for Letters}
\begin{center}
\begin{tabular}{||c c c c c||} 
 \hline
 A & B & C & D & E  \\ [0.5ex] 
 \hline
 00000 & 00001 & 00010 & 00011 & 00100  \\ 
 \hline \hline
 F & G & H & I & J   \\ [0.5ex] 
 \hline
 00101 & 00110 & 00111 & 01000 & 01001  \\ 
 \hline \hline 
 K & L & M & N & O \\[0.5ex] 
 \hline
 01010 & 01011 & 01100 & 01101 & 01110 \\
 \hline \hline
 P & Q & R & S & T   \\ [0.5ex] 
 \hline
 01111 & 10000 & 10001 & 10010 & 10011 \\ 
 \hline \hline 
 U & V & W & X & Y \\ [0.5ex] 
 \hline 
10100 & 10101 & 10110 & 10111  & 11000 \\
 \hline \hline 
 Z & & & & \\ [0.5ex] 
 \hline
 11001 & & & &\\ 
 \hline 

\end{tabular}
\end{center}
\end{multicols}



\subsection*{RSA Recovery Exponent:}
Given a key $(n,k)$ we have a recovery exponent $j$ such that $\exists ! \quad j \in \mathbb{Z}$ s.t. 
\begin{equation} \notag 
    kj\equiv 1 \mod (\phi(n)).
\end{equation}

\subsection*{Phi Function}
Formulas for $\phi(n), n\in \mathbb{Z}$. 
$$\phi(n) \quad \text{If n is a prime then, } \phi(n)=n-1.$$
If n is not prime then we factor so that n is in its reduced prime factorization form and use, 
$$\phi(n)=\phi(p_1p_2)= \phi(p_1)\phi(p_2) $$


\subsection*{Euclidean Algorithm}
\begin{align*}
a &= q_1 b + r_1 \quad & \text{where } 0 \le r_1 < b \\
b &= q_2 r_1 + r_2 \quad & \text{where } 0 \le r_2 < r_1 \\
r_1 &= q_3 r_2 + r_3 \quad & \text{where } 0 \le r_3 < r_2 \\
&\vdots \\
r_{k-2} &= q_k r_{k-1} + r_k \quad & \text{where } 0 \le r_k < r_{k-1} \\
r_{k-1} &= q_{k+1} r_k + 0
\end{align*}

\subsection*{Caesar Cipher}
$$(x+3)\mod 26$$

\subsection*{Vigenere Encryption}
$C \equiv x+$ corresponding keyword letter $\mod 26$
\subsection*{Vigenere Decryption}
$x \equiv C-$ corresponding keyword letter $\mod 26$

\subsection*{Elgamal Encryption}
Encipher: Public key $(p,r,a)$; private key $k$, random integer $j$ \\ 
Find $r^j\mod p$ and $a^j\mod p$ \\ 
$a^j = y \mod p \to $ encipher text

\subsection*{Knapsack Encryption}
Modulus $= m$, multiplier $=a$, private sequence $=ps$, Public key is a string of all the super-increasing numbers multiplied by the $a$ and then $\mod m$. \\ 
Encryption: message to binary, and each bit with the corresponding sequence with private key.  

\subsection*{Primitive Roots}
A number $g$ is called a \textbf{primitive root modulo $n$} if the smallest positive integer $k$ such that
\[
g^k \equiv 1 \pmod{n}
\]
is $k = \phi(n)$, where $\phi(n)$ is Euler's Phi function. 

Equivalently, $g$ is a primitive root modulo $n$ if the powers
\[
g,\, g^2,\, g^3,\, \dots,\, g^{\phi(n)}
\]
produce all integers relatively prime to $n$ modulo $n$.

\subsection*{Order Modulo n}
Let $a$ and $n$ be integers with $\gcd(a, n) = 1$. 
The \textbf{order of $a$ modulo $n$}, denoted by $\operatorname{ord}_n(a)$, 
is the smallest positive integer $k$ such that
\[
a^k \equiv 1 \pmod{n}.
\]
If no such integer $k$ exists, then $a$ is said to have no order modulo $n$.



\end{document}
